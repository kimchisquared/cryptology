%Init
\documentclass[16pt]{article}
\usepackage[utf8]{inputenc}
\usepackage{amsthm, amsmath, amssymb, color, graphicx, wrapfig} 
\begin{document}
\title{Graded Homework 4}
\author{Kimberly Mandery}
\date{10/24/2018} 
\maketitle

%Exercises
\begin{enumerate}
\item (20 points)
Using a cipher wheel, decrypt the following message, which was encrypted by rotating 1 clockwise for the first letter, then 2 clockwise for the second letter, etc.

%xjhrf tnzhm zgahi uetxz jnbwn utrhe pomdn bjmau gorfa oizoc c
\begin{center}
  \begin{tabular}{@{} lllll @{}}

    XJHRF& TNZHM & ZGAHI & UETXZ & JNBWN \\ 
    UTRHE & POMDN & BJMAU & GORFA & OIZOC \\ 
    C &  &  & &  \\ 
  \end{tabular}
\end{center}
%abcdefghijklmnopqrstuvwxyz \\
%shifts possible: 0-25\\
\begin{center}
	\begin{tabular}{@{} llll @{}}

	block XJHRF(shift 1-5)& whena  & block TNZHM(shift 6-10)& ngryc\\
	block ZGAHI(shift 11-15)&ountt & block UETXZ(shift 16-20)&enbef\\
	block JNBWN(shift 21-25)&oreyo & block UTRHE(shift 0-4)&uspea\\
	block POMDN(shift 5-9)&kifve   &block BJMAU(shift 10-14)&ryang\\
	block GORFA(shift 15-19)&ryanh & block OIZOC(shift 20-24)&undre\\
	block C(shift 25)&d&&\\
	\end{tabular}
\end{center}


\textit{"When angry, count ten before you speak. If very angry, an hundred"}


\vspace{5mm}
\item (20 points) 
Let $\{p_1,p_2,\dots, p_r\}$ be a set of prime numbers, and let
\[N=p_1p_2\cdots p_r+1.\]
Prove that $N$ is divisible by some prime not in the original set.  Use this fact to deduce that there must be infinitely many prime numbers.
(This proof of the infinitude of primes appears in Euclid's \emph{Elements}.  Prime numbers have been studied for thousands of years.)

\textbf{Proof}: Suppose N can be written as a product of primes in the set $\{p_1,p_2,\dots p_j\}$ such that \[N=p_1p_2\dots p_j\]. We can observe that \[N-1=p_1p_2\cdots p_r\] from $N=p_1p_2\cdots p_r+1$. Since two consecutive numbers only differ by 1, there is no prime we can multiply $N-1$ by in order to obtain  $p_1,p_2,\dots p_j$. Thus, there must exist a $p_j* \in \{p_1,p_2,\dots p_j\}$ where $p_j* \not\in \{p_1,p_2,\dots, p_r\} $. This works for both finite and infinite sets of primes.

\vspace{5mm}
\item (20 points) 
Find all values of $x$ between 0 and $m-1$ that are solutions of the following congruences.
\begin{enumerate}
\item $x+17\equiv 23\pmod{37}$\\%6
$x\equiv (23-17)\pmod{37}$\\
$x \equiv 6 \pmod{37}\\$
The solution is $x=6$.\\

\item $x^2\equiv 3\pmod{11}$\\%56
$5^2\equiv 3\pmod{11}$\\
$6^2\equiv 3\pmod{11}$\\
The solutions $x=5,6$ were found by trial and error.\\

\item $x^2\equiv 2\pmod{13}$\\%DNE
No solutions in the interval from 0 to 13 give such an equivalency.\\


\item Find a single value $x$ that simultaneously solves the two congruences%31
\[x\equiv 3\pmod{7}\hspace{5cm} x\equiv 4\pmod{9}.\]
[Hint Note that every solution of the first congruence looks like $x=3+7y$ for some $y$.  Substitute this into the second congruence and solve for $y$; then use that to get $x$.]\\

Using the hint from above and the Euclidean Algorithm, we can use x=9 and y=7 to get $9\equiv 7(1)+2$. We can repeat the algorithm using x=7 and y=2 to obtain $7 \equiv 2(3)+1$. Since we obtained 1 for a remainder we can stop. Rearranging both we obtain $2\equiv 9-7(1)$ and $1\equiv 7-2(3)$.
Combining these two equations, we get $1\equiv 7-(9-7(1))(3)$ which is equal to $7+9(-3)+7(3)$. Combining like terms, we get $7(4)+9(-3)$. To finish, we must consider which mod to use and can do the following by using the Chinese Remainder Theorem. Since we combined 7 and 9, the mod is now $7*9$, or $63$. The final combination is
$4(7*4)+3(9*-3)mod(9*7)\equiv 112-81mod(63) \equiv 31 $

\end{enumerate}
\end{enumerate}
\end{document}
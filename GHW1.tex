%Junk
\documentclass[16pt]{article}
\usepackage[utf8]{inputenc}
\usepackage{amsthm} 
\usepackage{amsmath}
\usepackage{amssymb}
\usepackage{color} 
\usepackage{graphicx} 
\newtheorem{defn}{Defn} 
\newtheorem{prop}{Proposition} 
\newtheorem{thm}{Theorem}
\begin{document}
\title{Homework I}
\author{Kimberly Mandery}
\date{due 8/7/2018} 
\maketitle
%Questions
\begin{enumerate}
%Question 1
\item (20 points) In the following diagram, a piece of paper is first folded into thirds. By preforming the origami move of folding two points onto two lines, we obtain the picture below. Prove that $x=\sqrt[3]{2}$
%Graphic
\begin{center}
\includegraphics[scale=1]{./sqrt2}
\end{center}
First, we can use the Pythagorean theorem and $\Delta BAC$ to solve for $d$.
\[\overline{\rm AB}^2 +\overline{\rm BC}^2=\overline{\rm AC}^2 \\
\Rightarrow 1^2+d^2=(x+1-d)^2\\
\Rightarrow d=\frac{x^2+2x}{2x-2}\]
Next, we can find $\overline{\rm DA}$. 
\[x=\overline{\rm DA}+\frac{x+1}{3}
\Rightarrow\overline{\rm DA}=x-\frac{x+1}{3}
\Rightarrow\overline{\rm DA}=\frac{2x-1}{3}\]
We know $\overline{\rm AD}\perp\overline{\rm DE}$ and $\overline{\rm AC}\perp\overline{\rm AE} \Rightarrow \angle DEA \cong \angle BAC \Rightarrow \Delta DEA \cong \Delta BAC$.
\[\Rightarrow\frac{\overline{\rm BC}}{\overline{\rm AC}} = \frac{\overline{\rm DA}}{\overline{\rm EA}}
\Rightarrow \frac{d}{x+1-d}=\frac{2x-1}{x+1}
\Rightarrow\frac{x^2+2x}{x^2+2x+2}=\frac{2x-1}{x+1}\]
\[\Rightarrow x^3+3x^2+2x=2x^3+3x^2+2x-2
\Rightarrow x^3 = 2 \Rightarrow x=\sqrt[3]{2}.\]
%Question 2
\raggedright
\item(20 points) Write a proof to solve the equation $ax^2+bx+c=0$ for $x$. Explain each step, starting with the following equation\\
%Proof2
\[ax^2+bx+c=0\]
Subtract $c$ from both sides
\[ax^2+bx=-c\]
Divide both sides by $a$
\[x^2+\frac{b}{a}x=\frac{-c}{a}\]
Add $\frac{b^2}{4a^2}$ to both sides
\[x^2+\frac{b}{a}x+\frac{b^2}{4a^2}=\frac{-c}{a}+\frac{b^2}{4a^2}\]
Completing the square we get
\[(x+\frac{b}{2a})^2=\frac{-c}{a}+\frac{b^2}{4a^2}\]
Find a common denominator on the right hand side by multiplying $\frac{-c}{a}$ by $4a$.
\[(x+\frac{b}{2a})^2=\frac{-4ac}{4a^2}+\frac{b^2}{4a^2}\]
Now get rid of the power on the left by square rooting both sides.
\[x+\frac{b}{2a}=\pm\sqrt{\frac{-4ac}{4a^2}+\frac{b^2}{4a^2}}\]
Subtract off $-\frac{b}{2a}$ from both sides.
\[x=\pm\frac{\sqrt{b^2-4ac}}{2a}-\frac{b}{2a}\]
Rearranging we get
\[x=\frac{-b\pm\sqrt{b^2-4ac}}{2a}\]



\raggedright
%Question3
\item(20 points) Show that the following complex numbers are algebraic over $\mathbb{Q}$.\\
\begin{defn}[Algebraic]
Given a number b, b is said to be algebraic over the set $\mathbb{Q}$ if $\exists$ a polynomial of the form 
\[a_nx^n + a_{n-1}x^{n-1} + \cdots + a_1x +a_0 = 0 \] with non-zero coefficients $a_i \in \mathbb{Q}$ where b is the solution (i.e. $f(b)=0$).
\end{defn}
\begin{enumerate}
%parta: root2
\item $\sqrt{2}$\\
Let $x=\sqrt{2}$\\$x^2=2$\\$x^2-2=0$\\
To verify, we plug in $f(\sqrt{2})=(\sqrt{2})^2-2=2-2=0$.\\By Defn 1, since $x=\sqrt{2}$ solves the function $f(x)=x^2-2,$ 
\\$\sqrt{2}$ is algebraic.\\
%partb: rootn
\item $\sqrt{n}$ for $n \in \mathbb{Z}$\\Let $x=\sqrt{n}$\\$x^2=n$\\$x^2-n=0$\\By Defn 1, since $x=\sqrt{n}$ solves the function $f(x)=x^2-n,$
\\$\sqrt{n}$ is algebraic.\\
%partc:root3+root5
\item $\sqrt{3}+\sqrt{5}$\\Let $x=\sqrt{3}+\sqrt{5}$\\
Then $x^2=(\sqrt{3}+\sqrt{5})^2=8+2\sqrt{15}$\\
and $x^4=(\sqrt{3}+\sqrt{5})^4=124+32\sqrt{15}$\\
We can write each $x^n$ in terms of linear combinations of its coefficients.
\[x^4=124+32x_1\]
\[x^2=8+2x_1\]
\[x^4-16x^2=-4+0x_1\]
\[x^4-16x^2+4=0\]
Since $x=\sqrt{3}+\sqrt{5}$ solves the polynomial $x^4-16x^2+4$,\\
$x=\sqrt{3}+\sqrt{5}$ is algebraic.\\
%partd:cuberoot2+root2
\item$\sqrt[3]{2}+\sqrt{2}$\\
Let $x=\sqrt[3]{2}+\sqrt{2}$. We can write each $x^n$ as a linear combination of its roots. The following equations gives the corresponding coefficients for each subsequent power of x.\\
\[x^6=12+24X_1+60x_2+80x_3+60x_4+24x_5\]
\[x^4=4+2x_2+8x_3+12x_4+8x_5\]
\[x^3=2+6X_1+6x_2+2x_3\]
\[x^2=2+1x_4+2x_5\]
The work is left to the reader, but we should get our $f(x)=x^6-6x^4-4x^3+12x^2-24x-4$. Since $\exists f(x)$ where $f(\sqrt{3}+\sqrt{5})$ is equal to zero, $x=\sqrt[3]{2}+\sqrt{2}$ is algebraic.

\end{enumerate}
\end{enumerate}
\end{document}

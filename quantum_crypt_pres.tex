\documentclass{beamer}
\usepackage[utf8]{inputenc}
\usepackage[latin1]{inputenc}
\usepackage{graphicx}	 			% insert PostScript figures
\usepackage{amsmath}
\usepackage{amssymb}
\usepackage{textcomp}
\usepackage{wrapfig}
\usepackage{textcomp}
\usepackage{subfigure}
\usepackage{multicol}
\usepackage{amsthm}
\usepackage{mathtools}
\usepackage{enumerate}
\usepackage{relsize}
\pagestyle{empty}
\usepackage[top=.9in, bottom=.9in, left=.6in, right=.6in]{geometry}
\usepackage{indentfirst}
\usepackage{url}
\usepackage{cite}
\usepackage{xcolor}
\usepackage{biblatex}
\usetheme{Copenhagen}
\usecolortheme{seahorse}

\title[Quantum Cryptography]{Quantum Cryptography}
\author{Jordan Klumper, Kimberly Mandery}
\institute{MATH5347: Cryptography}
\date{November 2018}
\begin{document}
\frame{\titlepage}
 
%Slide 2
\begin{frame}
\frametitle{Quantum Computing: Qubit}
\begin{center}
\includegraphics[scale=.8]{qubit.png}
\end{center}
\end{frame}
%Slide 2
\begin{frame}
\frametitle{Quantum Computing: Superposition}
\begin{center}
Classical Bit vs Quantum Bit
\includegraphics[scale=.5]{superposition.png}
\end{center}
\end{frame}

\begin{frame}
\frametitle{Quantum Computing: Superposition}
\begin{center}
\includegraphics[scale=0.35]{supercoin.png}
\end{center}
\end{frame}

%Slide 2
\begin{frame}
\frametitle{Quantum Computing: Entanglement}
\begin{center}
\includegraphics[scale=.8]{entanglecoin.jpg}
\end{center}
\end{frame}

%Example
\begin{frame}
\frametitle{Quantum Key Exchange}
\centering
\includegraphics[scale=.5]{setup1.PNG}\\
\includegraphics[scale=.5]{setup2.PNG}
\end{frame}
\begin{frame}
\frametitle{Quantum Key Exchange}
\centering
\includegraphics[scale=.5]{alicestep1.PNG}\\ 
\includegraphics[scale=.5]{alicestep2.PNG}
\end{frame}
\begin{frame}
\frametitle{Quantum Key Exchange}
\centering
\includegraphics[scale=.5]{bobstep3.PNG}
\end{frame}
\begin{frame}
\frametitle{Quantum Key Exchange}
\centering
\includegraphics[scale=.5]{bobalicecomparestep4.PNG}
\end{frame}
\begin{frame}
\frametitle{Quantum Key Exchange}
\centering
\includegraphics[scale=.5]{bobalicetalkstep5.png}
\end{frame}
\begin{frame}
\frametitle{Quantum Key Exchange}
\centering
\includegraphics[scale=.5]{step6.PNG}\\
\href{https://www.youtube.com/watch?v=zL4HSk4MUUw}{\beamergotobutton{Human Tetris}}
\end{frame}


\begin{frame}
\frametitle{Quantum Key Exchange}
\begin{figure}[h]
\centering
\includegraphics[scale=1]{examplelab3.png}
\end{figure}
\end{frame}
\begin{frame}{Quantum Key Exchange}
\begin{figure}[h]
\centering
\includegraphics[scale=.85]{examplelab2.png}
\end{figure}
\end{frame}
%Slide 3
\begin{frame}
\frametitle{Why does this matter in Cryptography?}
Most used systems potentially insecure\\
Shor's Algorithm
\end{frame}

\begin{frame}
\frametitle{Shor's Algorithm}
    \begin{enumerate}
        \item Pick a number $a < N$ \\
    where $a$ not a factor of $N$. \pause
        \item Find $r$ such that $a\text{ mod N}$ \\
    where $a^r \equiv 1\text{ mod N}$, \\
    this is called the period or order \pause
        \item Check: $r$ is even \\
    and $a^{1/2}+1 \not \equiv 1 \text{ mod N}$ \pause
        \item Then $p=gcd(a^{1/2}-1, N)$ \\
    and $q=gcd(a^{1/2}+1, N)$ 
    \end{enumerate}
\end{frame}
\begin{frame}{Shor's Algorithm}
    \begin{enumerate}
        \item Quantum Fourier Transform : uses resonances to amplify the basic state associated with the correct period and suppress amplitudes that incorrectly interfere.\pause
        \item Complex Analysis (i.e. complex roots of unity) : can then be used in order to find most probabilistic space.
    \end{enumerate}
\end{frame}


%Slide 4.a.1
\begin{frame}
\frametitle{AES}
{\bf AES - Advanced Encrypton Standard}\\
{\bf What is AES?}\\
A single key (symmetric key) is used in both encrypting and decryption. Messages are split into 128-bit size pieces that are turned into matrices.\\
\begin{figure}[h]
\caption{AES Schematic}
\centering
\includegraphics[scale=0.3]{aes.jpg}
\end{figure}
\end{frame}

%Slide 4.a.2
\begin{frame}
\frametitle{AES}
{\bf How is it broken?}\\
\begin{itemize}
\item
Shor's Algorithm is not helpful\\
\pause
\item
Grover's Algorithm (another quantum algorithm) which allows the computation time to break AES to go from $\mathcal{O}(n)$ in classical to $\mathcal{O}(\sqrt{n})$ in quantum computing\\
\includegraphics[scale=.35]{grover.png}
\end{itemize}
\end{frame}
\begin{frame}
\frametitle{AES}
{\bf How is it broken?}\\
\begin{itemize}
\item
Shor's Algorithm is not helpful\\
\item
Grover's Algorithm (another quantum algorithm) which allows the computation time to break AES to go from $\mathcal{O}(n)$ in classical to $\mathcal{O}(\sqrt{n})$ in quantum computing\\
\end{itemize}
\vspace{0.1in}
{\bf How is it made resistant to quantum computing?}\\
\begin{itemize}
\item
Choose largest keys possible, increases the time it takes. 
\item
Use authentication systems which offers additional layers to the security of the system.
\end{itemize}
\end{frame}

%Slide 4.b.1
\begin{frame}
\frametitle{RSA}
{\bf RSA - Rivest-Shamir-Adleman}\\
{\bf What is RSA?}\\
An asymmetrical cryptosystem that utilizes the discrete logarithm problem, that is it's difficult to factor the product of two very large primes\\
\begin{enumerate}
    \item Choose two large similar sized primes $p$ and $q$.
    \item Find $n=p\cdot q$ which is our modulus
    \item Find $\phi(n)=\phi(p)\cdot\phi(q)=(p-1)(q-1)$
    \item Given $e$, the encryption key, find $d$ where $d\equiv e^{-1}\pmod{\phi(n)}$
    \item We now have our public key: $(n,e)$ and our private key: $(d,p,q)$
\end{enumerate}\\
\end{frame}

%Slide 4.b.2
\begin{frame}
\frametitle{RSA}
{\bf How is it broken?}\\
\begin{itemize}
    \pause
    \item Given our public key: $(n,e)$, Shor's algorithm allows us to find $p$ and $q$
    \pause
    \item With $p$ and $q$ we can readily find $d$ using step 4 on the previous slide now that $\phi(n)$ is easily computed
\end{itemize}
\vspace{0.1in}
{\bf How is it made resistant to quantum computing?}\\
\begin{itemize}
    \pause
    \item As of right now there aren't any widely accepted ways to improve
    RSA
    \pause
    \item XMSS (Extended Merkle Signature Scheme) has been suggested as a replacement as it's resistant to quantum algorithms.
    \pause
    \item Rather than a discrete logarithm problem, XMSS uses a hash-based function for security, these are considered quantum resistant
\end{itemize}
\end{frame}

%Slide 4.c.1
\begin{frame}
\frametitle{ECC}
{\bf ECC - Elliptical Curve Cryptography}\\
{\bf What is ECC?}\\
A cryptosystem that uses an algebraic structure created by points on an elliptic curve over a finite field. It's another system based on the discrete logarithm problem.
\begin{center}
    Elliptic curve form: $Y^2=X^3+AX+B$ where $4A^3 + 27B^2 \ne 0$\\
    Discrete logarithm problem: Give points $P$ and $Q$ find $n$ such that $Q=nP$
\end{center}
\end{frame}

%Slide 4.c.2
\begin{frame}
\frametitle{ECC}
{\bf How is it broken?}\\
\begin{itemize}
    \pause
    \item Similar to RSA, Shor's algorithm can be used to find $n$ and allowing an eavesdropper to determine the private key
    \pause
\end{itemize}
\vspace{0.1in}
{\bf How is it made resistant to quantum computing?}\\
\begin{itemize}
    \pause
    \item  Supersingular  Isogeny  Key  Exchange  is a direct  improvement to ECC
\end{itemize}
\pause
$$j(E)=1728\frac{4A^3}{4A^3+27B^2}$$
\begin{itemize}
\pause
    \item If the $j$-invariant of two curves is the same, they are isomorphic over the same field.
\end{itemize}
\pause
\begin{center}
 Alice finds $\phi_A(E)$ sends it to Bob who finds  $\phi_B(E_A)$ which is $E_{AB}$
 Now Bob has $E_{AB}$, similar to above Alice ends up with $E_{BA}$\\
 $E_{AB}\cong E_{BA}$
\end{center}
%Two parties begin with a common public curve and each create a new curve both of which are isomorphic
%Security comes from the mappings being difficult to find, as well as the time it would take Shor's algorithm to find $n$ of ECC
\end{frame}

%Slide 5
\begin{frame}
\frametitle{When will this occur?}
\begin{itemize}
\pause
\item As of 2018 Google has created a 72-qubit quantum chip, the largest yet.
\pause
\item To even begin writing software on quantum computers, hundreds to thousands of qubits would be needed. 
\pause
\item To run Shor's algorithm on a 2,000-bit number a quantum computer is estimated to require 130 million cubits.
\pause
\item For quantum cryptography, estimates say we are 10 to 20 years away, but estimates vary widely.
\end{itemize}
\end{frame}

%Slide 6
\begin{frame}
\frametitle{Thanks}

\includegraphics[scale=.25]{smile.PNG}
\end{frame}
%Slide 6
\begin{frame}
\frametitle{Thanks}

\includegraphics[scale=1]{smile.PNG}
\end{frame}
%Slide 6
\begin{frame}
\frametitle{Thanks}

\includegraphics[scale=1.75]{smile.PNG}
\end{frame}
%Slide 6
\begin{frame}
\frametitle{Thanks}
\includegraphics[scale=2.75]{smile.PNG}
\end{frame}
%Slide 6
\begin{frame}
\frametitle{Thanks}
\includegraphics[scale=4]{smile.PNG}
\end{frame}
%Slide 6
\begin{frame}
\frametitle{Thanks}
\includegraphics[scale=5.25]{smile.PNG}
\end{frame}
%Slide 7


\begin{thebibliography}{1}
  \begin{frame}
 \frametitle{Resources}
  \bibitem{entangle} {\em Visualizing 2-Qubit Entanglement}\\ \textt{http://algassert.com/post/1716}
  
  \bibitem{quantumdef} {\em Quantum Theory}\\
  \textt{https://whatis.techtarget.com/definition/quantum-theory}
  
  \bibitem{quantumerror} {\em Quantum Error Correction}\\
  \textt{https://en.wikipedia.org/wiki/quantum-error-correction}
  
  \bibitem{feynman} {\em Richard Feynman Talks}\\
  \textt{https://en.wikiquote.org/wiki/Talk:Richard_Feynman}
  
  \bibitem{superpos1} {\em Superposition Double-Slit Paradox}\\
  \textt{https://phys.org/news/2014-10-superposition-revisited-resolution-double-slit-paradox.html}
  
  \bibitem{example} {\em Alice and Bob Lab Example}\\
  \textt{https://ieeexplore.ieee.org}

\end{frame}

\begin{frame}
\frametitle{Resources}
  \bibitem{shor} {\em Complexity of Classical vs Shor's Algorithm}\\ \textt{https://quantumexperience.ng.bluemix.net/proxy/tutorial/full-user-guide/004-Quantum\_Algorithms/110-Shor\%27s\_algorithm.html}
  
  \bibitem{crypto} Jeffrey Hoffstein, Jill Pipher, J.H. Silverman {\em An Introduction to Mathematical Cryptography} 2008: Springer-Verlag New York
  
  \bibitem{today} {\em How secure is today's encryption against quantum computers?}\\ \texttt{https://betanews.com/2017/10/13/current-encryption-vs-quantum-computers/}
  
  \bibitem{aes} {\em Advanced Encryption Standard}\\ \texttt{https://en.wikipedia.org/wiki/Advanced_Encryption_Standard}
  
  \bibitem{grover} {\em Grover's Algorithm}\\ \texttt{ https://en.wikipedia.org/wiki/Grover\%27s\_algorithm}
  
  \bibitem{hash} {\em Hash-based cryptography}\\ \texttt{https://en.wikipedia.org/wiki/Hash-based_cryptography}
  
  \end{frame}

\begin{frame}
\frametitle{Resources}
  \bibitem{quantum1} {\em Quantum Computing Lecture}\\ \texttt{https://people.eecs.berkeley.edu/~luca/quantum/lecture02.pdf}

  \bibitem{tutpoint} {\em Advanced Encryption Standard}\\
  \textt{https://www.tutorialspoint.com/cryptography/advanced\_encryption\_standard.htm}
  
  \bibitem{kerb} {\em Kerberos (protocol)}\\
  \textt{https://en.wikipedia.org/wiki/Kerberos\_(protocol)}
 
  \bibitem{ecc1} {\em Overview of History of Elliptic Curves and its
  use in cryptography}\\ \textt{https://www.ijser.org/researchpaper/Overview-of-History-of-Elliptic-Curves-and-its-use-in-cryptography.pdf}


  \bibitem{resist} {\em The 256-bit AES Resists Quantum Attacks}\\
  \textt{https://www.researchgate.net/publication/316284124\_The\_AES-256\_Cryptosystem\_Resists\_Quantum\_\\Attacks}
  
  \bibitem{RSA} {\em RSA (cryptosystem)}\\
  \textt{https://en.wikipedia.org/wiki/RSA\_(cryptosystem)}
  
  \bibitem{tools} {\em XMSS: Extended Hash-Based Signatures}\\
  \textt{https://tools.ietf.org/id/draft-irtf-cfrg-xmss-hash-based-signatures-10.html}
  
  \bibitem{andrea} {\em Elliptic Curve Cryptography: ECDH and ECDSA}\\
  \textt{http://andrea.corbellini.name/2015/05/30/elliptic-curve-cryptography-ecdh-and-ecdsa/}
  
   \bibitem{ECC} {\em Supersingular isogeny key exchange}\\
  \textt{https://en.wikipedia.org/wiki/Supersingular\_isogeny\_key\_exchange}
  
  \bibitem{osa} {\em Quantum Computing: How Close Are We?}\\ \textt{https://www.osa-opn.org/home/articles/volume\_27/october\_2016/features/quantum\_computing\_
  \\how\_close\_are\_we/}
  
  \bibitem{sa} {\em How Close Are We—Really—to Building a Quantum Computer?}\\ \textt{https://www.scientificamerican.com/article/how-close-are-we-really-to-building-a-quantum-computer/}
  
  \bibitem{time} {\em Timeline of quantum computing}\\ \textt{https://en.wikipedia.org/wiki/Timeline_of_quantum_computing}
\end{frame}
\end{thebibliography}
\begin{frame}{Questions?}
\centering
    \includegraphics[scale=0.2]{qcomp.png}
\end{frame}
\end{document}
\documentclass[12pt]{article}
%\documentclass[12pt,a4paper]{article}
%\documentclass[aip,amsmath,amssymb,reprint]{revtex4-1}
%\documentclass[aip,amsmath,amssymb,graphicx]{revtex4-1}

\usepackage[latin1]{inputenc}
\usepackage{graphicx}	 			% insert PostScript figures
\usepackage{amsmath}
\usepackage{amssymb}
\usepackage{textcomp}
\usepackage{wrapfig}
\usepackage{textcomp}
\usepackage{subfigure}
\usepackage{multicol}
\usepackage{amsmath}
\usepackage{mathtools}
\usepackage{enumerate}
\usepackage{relsize}
\pagestyle{empty}
\usepackage[top=.9in, bottom=.9in, left=.6in, right=.6in]{geometry}
\usepackage{indentfirst}
\usepackage{url}
\usepackage{cite}
\pagestyle{plain}

\newcommand\md{\ }


\begin{document}              

\begin{centering}
	
	MATH 5347: APPLIED ALGEBRA WITH CRYPTOLOGY\\

 \line(1,0){500}

\end{centering}

	\par\vspace{3mm}

\begin{centering}
	{\bf Graded Homework 7 (40 points)}
	\par\vspace{1mm}
	{ Kimberly Mandery}
	\par\vspace{1mm}
	{\bf DUE 1600 19 November 2018}


\end{centering}



\vspace{3mm}

{\bf Instructions}: This is a graded homework assignment worth 40 points.  Solutions must be organized, neat in appearance, and must clearly indicate the answer to each problem.  Be sure to show your work where appropriate.  

\vspace{5mm}

\begin{enumerate}
\item (20 points)
Let $E$ be the elliptic curve 
\[E\colon y^2=x^3+x+1\]
and let $P=(4,2)$ and $Q=(0,1)$ be the points on $E$ modulo 5.  Solve the elliptic curve discrete logarithm problem for $P$ and $Q$, that is, find a positive integer $n$ such that $Q=nP$.

$\textbf{Step 1: }$ Our basic format is as follows:
\[nP = [x_3, y_3, 1] = [\lambda ^2 - x_1 - x_2, \lambda (x_1 - x_3) - y_1, 1 ]\]
\[\text{ where } nP = P_1 \oplus P_2 = [x_1,y_1,1]\oplus[x_2,y_2,1] \text{ and } \lambda = \begin{dcases} \frac{y_2-y_1}{x_2-x_1} & \text{if } $P_1\not = P_2$ \\\frac{3x_1^2+1}{2y_1} & \text{if } $P_1 = P_2$. \\\end{dcases}\]

$\textbf{Step 2: } $In order to compute $2P$, we must first compute the appropriate $\lambda$.\\
Beginning with $P_1 = P_2 = [4,2,1]$ we can first calculate $\lambda_{2P}$. \[\lambda_{2P} = \frac{3(4)^2+1}{2(2)} = \frac{48+1}{4} \equiv \frac{4}{4} \equiv 1.\]
Now we can compute $2P$ as \[2P = [1 ^2 - 4 - 4, 1 (4 - x_3) - 2, 1 ] \equiv [3,4,1].\]
We can repeat the process until we get $nP \equiv [0,1,1]$.\\

$\textbf{Step 3: }$ $3P\equiv P \oplus 2P$ where $P_1 \equiv [4,2,1]$ and $ P_2 \equiv [3,4,1]$.\\
\[\lambda_{3P}=\frac{4-2}{3-4}\equiv \frac{2}{-1}\equiv -2 \equiv 3 \]
\[3P= [3 ^2 - 4 - 3, 3 (4 - x_3) - 2, 1 ] \equiv [2,4,1].\]

$\textbf{Step 4: }$ $4P\equiv 2P \oplus 2P$ where $P_1 = P_2 = [3,4,1]$.\\
\[\lambda_{4P}=\frac{3(3)^2+1}{2(4)}\equiv \frac{3}{3}\equiv 1 \]
\[4P= [1 ^2 - 3 - 3, 1 (3 - x_3) - 2, 1 ] \equiv [0,4,1].\]

$\textbf{Step 5: }$ $5P\equiv 2P \oplus 3P$ where $P_1 \equiv [3,4,1]$ and $P_2 \equiv [2,4,1]$.\\
\[\lambda_{5P}=\frac{4-4}{2-3}\equiv \frac{0}{-1}\equiv 0 \]
\[5P= [0 ^2 - 3 - 2, 1 (3 - x_3) - 2, 1 ] \equiv [0,1,1].\]

thus \textbf{Q = 5P}.


\item (20 points)
Use the double-and-add algorithm (Table 6.3) to compute $nP$ in $E(\mathbb{F}_p)$ for the following curves and points.
\[E\colon Y^2=x^3+23X+13,\hspace{2cm} p=83,\hspace{1cm}P=(24,14),\hspace{1cm}n=19.\]

We use the method described in Step 1 of Exercise 1, but replace A=1 with A=23 in our $\lambda$ function. However, instead of having to do 19 calculations in total, we can use the double-and-add algorithm to compute $16P \oplus 3P \equiv 19P$. In order to make use of the inverse property, use Wolfram Alpha to calculate inverses.

$\textbf{Step 1: }$ $2P\equiv P \oplus P$ where $P_1 \equiv P_2 \equiv [24,14,1]$.\\
\[\lambda_{2P}=\frac{3(24)^2+23}{2(14)}\equiv \frac{1751}{28}\equiv \frac{8}{28} \equiv 8*28^{-1} \equiv 8*3 \equiv 24 \text{ mod 83}\]
\[2P= [24 ^2 - 24 - 24, 24 (24 - 30) - 14, 1 ] \equiv [30,8,1].\]

$\textbf{Step 2: }$ $4P\equiv 2P \oplus 2P$ where $P_1 \equiv P_2 \equiv [30,8,1]$.\\
\[\lambda_{4P}=\frac{3(30)^2+23}{2(8)}\equiv \frac{2723}{16}\equiv \frac{67}{16} \equiv 67*16^{-1} \equiv 67*26 \equiv 82 \text{ mod 83}\]
\[4P= [82 ^2 - 30 - 30, 82 (30 - 24) - 8, 1 ] \equiv [24,69,1].\]

$\textbf{Step 3: }$ $8P\equiv 4P \oplus 4P$ where $P_1 \equiv P_2 \equiv [24,69,1]$.\\
\[\lambda_{8P}=\frac{3(24)^2+23}{2(69)}\equiv \frac{1751}{138}\equiv \frac{8}{55} \equiv 8*55^{-1} \equiv 8*80 \equiv 59 \text{ mod 83}\]
\[8P= [59 ^2 - 24 - 24, 59 (24 - 30) - 69, 1 ] \equiv [30,75,1].\]

$\textbf{Step 4: }$ $16P\equiv 8P \oplus 8P$ where $P_1 \equiv P_2 \equiv [30,75,1]$.\\
\[\lambda_{16P}=\frac{3(30)^2+23}{2(75)}\equiv \frac{2723}{150}\equiv \frac{67}{67} \equiv 1 \text{ mod 83}\]
\[16P= [1 ^2 - 30 - 30, 1 (30 - 24) - 75, 1 ] \equiv [24,14,1].\]

Since $16P\equiv P$, we can reduce our equation $19P\equiv 16P\oplus 3P\equiv P\oplus 3P\equiv 4P=[24,69,1]$

Thus, \textbf{19P $\equiv$ [24,69,1]}.
\end{enumerate}



\end{document}